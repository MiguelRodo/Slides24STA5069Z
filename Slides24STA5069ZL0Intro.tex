% Options for packages loaded elsewhere
\PassOptionsToPackage{unicode}{hyperref}
\PassOptionsToPackage{hyphens}{url}
\PassOptionsToPackage{dvipsnames,svgnames,x11names}{xcolor}
%
\documentclass[
  ignorenonframetext,
  aspectratio=169,
]{beamer}
\usepackage{pgfpages}
\setbeamertemplate{caption}[numbered]
\setbeamertemplate{caption label separator}{: }
\setbeamercolor{caption name}{fg=normal text.fg}
\beamertemplatenavigationsymbolsempty
% Prevent slide breaks in the middle of a paragraph
\widowpenalties 1 10000
\raggedbottom
\setbeamertemplate{part page}{
  \centering
  \begin{beamercolorbox}[sep=16pt,center]{part title}
    \usebeamerfont{part title}\insertpart\par
  \end{beamercolorbox}
}
\setbeamertemplate{section page}{
  \centering
  \begin{beamercolorbox}[sep=12pt,center]{part title}
    \usebeamerfont{section title}\insertsection\par
  \end{beamercolorbox}
}
\setbeamertemplate{subsection page}{
  \centering
  \begin{beamercolorbox}[sep=8pt,center]{part title}
    \usebeamerfont{subsection title}\insertsubsection\par
  \end{beamercolorbox}
}
\AtBeginPart{
  \frame{\partpage}
}
\AtBeginSection{
  \ifbibliography
  \else
    \frame{\sectionpage}
  \fi
}
\AtBeginSubsection{
  \frame{\subsectionpage}
}

\usepackage{amsmath,amssymb}
\usepackage{iftex}
\ifPDFTeX
  \usepackage[T1]{fontenc}
  \usepackage[utf8]{inputenc}
  \usepackage{textcomp} % provide euro and other symbols
\else % if luatex or xetex
  \usepackage{unicode-math}
  \defaultfontfeatures{Scale=MatchLowercase}
  \defaultfontfeatures[\rmfamily]{Ligatures=TeX,Scale=1}
\fi
\usepackage{lmodern}
\ifPDFTeX\else  
    % xetex/luatex font selection
\fi
% Use upquote if available, for straight quotes in verbatim environments
\IfFileExists{upquote.sty}{\usepackage{upquote}}{}
\IfFileExists{microtype.sty}{% use microtype if available
  \usepackage[]{microtype}
  \UseMicrotypeSet[protrusion]{basicmath} % disable protrusion for tt fonts
}{}
\makeatletter
\@ifundefined{KOMAClassName}{% if non-KOMA class
  \IfFileExists{parskip.sty}{%
    \usepackage{parskip}
  }{% else
    \setlength{\parindent}{0pt}
    \setlength{\parskip}{6pt plus 2pt minus 1pt}}
}{% if KOMA class
  \KOMAoptions{parskip=half}}
\makeatother
\usepackage{xcolor}
\newif\ifbibliography
\setlength{\emergencystretch}{3em} % prevent overfull lines
\setcounter{secnumdepth}{-\maxdimen} % remove section numbering


\providecommand{\tightlist}{%
  \setlength{\itemsep}{0pt}\setlength{\parskip}{0pt}}\usepackage{longtable,booktabs,array}
\usepackage{calc} % for calculating minipage widths
\usepackage{caption}
% Make caption package work with longtable
\makeatletter
\def\fnum@table{\tablename~\thetable}
\makeatother
\usepackage{graphicx}
\makeatletter
\def\maxwidth{\ifdim\Gin@nat@width>\linewidth\linewidth\else\Gin@nat@width\fi}
\def\maxheight{\ifdim\Gin@nat@height>\textheight\textheight\else\Gin@nat@height\fi}
\makeatother
% Scale images if necessary, so that they will not overflow the page
% margins by default, and it is still possible to overwrite the defaults
% using explicit options in \includegraphics[width, height, ...]{}
\setkeys{Gin}{width=\maxwidth,height=\maxheight,keepaspectratio}
% Set default figure placement to htbp
\makeatletter
\def\fps@figure{htbp}
\makeatother
\newlength{\cslhangindent}
\setlength{\cslhangindent}{1.5em}
\newlength{\csllabelwidth}
\setlength{\csllabelwidth}{3em}
\newlength{\cslentryspacingunit} % times entry-spacing
\setlength{\cslentryspacingunit}{\parskip}
\newenvironment{CSLReferences}[2] % #1 hanging-ident, #2 entry spacing
 {% don't indent paragraphs
  \setlength{\parindent}{0pt}
  % turn on hanging indent if param 1 is 1
  \ifodd #1
  \let\oldpar\par
  \def\par{\hangindent=\cslhangindent\oldpar}
  \fi
  % set entry spacing
  \setlength{\parskip}{#2\cslentryspacingunit}
 }%
 {}
\usepackage{calc}
\newcommand{\CSLBlock}[1]{#1\hfill\break}
\newcommand{\CSLLeftMargin}[1]{\parbox[t]{\csllabelwidth}{#1}}
\newcommand{\CSLRightInline}[1]{\parbox[t]{\linewidth - \csllabelwidth}{#1}\break}
\newcommand{\CSLIndent}[1]{\hspace{\cslhangindent}#1}

\usepackage{xcolor}
\usepackage{amssymb} % For additional symbols

% Definitions for all colors used in the presentation
\definecolor{balancedOrange}{RGB}{255,216,180}
\definecolor{secondaryOrange}{RGB}{251,188,5}
\definecolor{softBlue}{RGB}{100,124,156}
\definecolor{titleColor}{RGB}{0,51,102}
\definecolor{accentColor}{RGB}{112,173,71}

% Applying colors to Beamer elements
\setbeamercolor{palette primary}{bg=balancedOrange,fg=titleColor}
\setbeamercolor{palette secondary}{bg=secondaryOrange,fg=white}
\setbeamercolor{palette tertiary}{bg=softBlue,fg=white}
\setbeamercolor{titlelike}{parent=palette primary, fg=titleColor}
\setbeamercolor{structure}{fg=softBlue}
\setbeamercolor{block title}{bg=balancedOrange,fg=titleColor}
\setbeamercolor{block body}{bg=balancedOrange!10,fg=black}

% Customizing bullet points to a smaller sideways triangle and adjusting vertical position
\setbeamertemplate{itemize item}{\raisebox{1.5pt}{\scalebox{0.6}{$\blacktriangleright$}}}
\setbeamertemplate{itemize subitem}{\raisebox{1.3pt}{\scalebox{0.5}{$\blacktriangleright$}}}
\setbeamertemplate{itemize subsubitem}{\raisebox{1.1pt}{\scalebox{0.4}{$\blacktriangleright$}}}

% Suppressing the footline on the title slide
\setbeamertemplate{footline}{
    \ifnum\insertpagenumber=1
    \else
        \hfill\insertframenumber/\inserttotalframenumber\hspace*{1em}\vspace*{1em}
    \fi
}

% Must 
\makeatletter
\makeatother
\makeatletter
\makeatother
\makeatletter
\@ifpackageloaded{caption}{}{\usepackage{caption}}
\AtBeginDocument{%
\ifdefined\contentsname
  \renewcommand*\contentsname{Table of contents}
\else
  \newcommand\contentsname{Table of contents}
\fi
\ifdefined\listfigurename
  \renewcommand*\listfigurename{List of Figures}
\else
  \newcommand\listfigurename{List of Figures}
\fi
\ifdefined\listtablename
  \renewcommand*\listtablename{List of Tables}
\else
  \newcommand\listtablename{List of Tables}
\fi
\ifdefined\figurename
  \renewcommand*\figurename{Figure}
\else
  \newcommand\figurename{Figure}
\fi
\ifdefined\tablename
  \renewcommand*\tablename{Table}
\else
  \newcommand\tablename{Table}
\fi
}
\@ifpackageloaded{float}{}{\usepackage{float}}
\floatstyle{ruled}
\@ifundefined{c@chapter}{\newfloat{codelisting}{h}{lop}}{\newfloat{codelisting}{h}{lop}[chapter]}
\floatname{codelisting}{Listing}
\newcommand*\listoflistings{\listof{codelisting}{List of Listings}}
\makeatother
\makeatletter
\@ifpackageloaded{caption}{}{\usepackage{caption}}
\@ifpackageloaded{subcaption}{}{\usepackage{subcaption}}
\makeatother
\makeatletter
\@ifpackageloaded{tcolorbox}{}{\usepackage[skins,breakable]{tcolorbox}}
\makeatother
\makeatletter
\@ifundefined{shadecolor}{\definecolor{shadecolor}{rgb}{.97, .97, .97}}
\makeatother
\makeatletter
\makeatother
\makeatletter
\makeatother
\ifLuaTeX
  \usepackage{selnolig}  % disable illegal ligatures
\fi
\IfFileExists{bookmark.sty}{\usepackage{bookmark}}{\usepackage{hyperref}}
\IfFileExists{xurl.sty}{\usepackage{xurl}}{} % add URL line breaks if available
\urlstyle{same} % disable monospaced font for URLs
\hypersetup{
  pdftitle={Introduction to STA5069Z},
  pdfauthor={Miguel Rodo},
  colorlinks=true,
  linkcolor={blue},
  filecolor={magenta},
  citecolor={Blue},
  urlcolor={cyan},
  pdfcreator={LaTeX via pandoc}}

\title{Introduction to STA5069Z}
\subtitle{Modern multivariate statistical techniques}
\author{Miguel Rodo}
\date{2024-02-12}

\begin{document}
\frame{\titlepage}
\ifdefined\Shaded\renewenvironment{Shaded}{\begin{tcolorbox}[boxrule=0pt, sharp corners, interior hidden, frame hidden, enhanced, breakable, borderline west={3pt}{0pt}{shadecolor}]}{\end{tcolorbox}}\fi

\begin{frame}{Welcome to STA5069Z}
\protect\hypertarget{welcome-to-sta5069z}{}
\begin{itemize}
\tightlist
\item
  This course follows on from the honours module in multivariate
  statistics. It is aimed at looking at the application of multivariate
  techniques to modern-day data problems, including big data problems.
  Techniques introduced at honours level will be reviewed and extended
  and new techniques will be introduced.

  \begin{itemize}
  \tightlist
  \item
    See
    \href{https://sebnemer.github.io/english/courses/multivariate/}{course
    website} for specific topics coverered.
  \end{itemize}
\item
  \textbf{Lecturers}

  \begin{itemize}
  \tightlist
  \item
    Mr Miguel Rodo (PD Hahn 5.52)
  \item
    Dr Sebnem Er (PD Hahn 5.55)
  \end{itemize}
\end{itemize}

\addtocounter{framenumber}{-1}
\end{frame}

\begin{frame}{Key references}
\protect\hypertarget{key-references}{}
\begin{itemize}
\tightlist
\item
  Primary: Izenman (2008)

  \begin{itemize}
  \tightlist
  \item
    Available (for purchase) at
    \href{https://www.springer.com/gp/book/9780387781884}{Springer} and
    \href{https://www.amazon.com/Modern-Multivariate-Statistical-Techniques-Classification/dp/0387781889}{Amazon}
  \end{itemize}
\item
  Also used: Hastie, Tibshirani, and Friedman (2009)

  \begin{itemize}
  \tightlist
  \item
    Available for free at
    \href{https://hastie.su.domains/ElemStatLearn/}{Hastie's own
    website}
  \end{itemize}
\item
  Paper references will be provided as we go
\end{itemize}
\end{frame}

\begin{frame}{Course structure: lectures}
\protect\hypertarget{course-structure-lectures}{}
\begin{itemize}
\tightlist
\item
  The
  \href{https://sebnemer.github.io/english/courses/multivariate/}{course
  website} has all the information you need.
\end{itemize}

\textbf{Format}

\begin{itemize}
\tightlist
\item
  Each lecture will cover a new set of multivariate techniques.
\item
  These is a \emph{lot} of content, which will not all be covered in
  lectures.
\item
  As such, significant self study is required.

  \begin{itemize}
  \tightlist
  \item
    Lecture slides and associated recommended readings will be provided
    in advance.
  \end{itemize}
\item
  During lectures, we will cover key concepts, tackle student questions
  and review tutorial solutions.
\end{itemize}

\textbf{Venue and timing}

\begin{itemize}
\tightlist
\item
  Lectures will be 4-5.45pm on Mondays and Wednesdays in PD Hahn 4.26
  (the Quiet Room).
\end{itemize}
\end{frame}

\begin{frame}[fragile]{Course structure: practice and testing}
\protect\hypertarget{course-structure-practice-and-testing}{}
\textbf{Tutorials}

\begin{itemize}
\tightlist
\item
  For each lecture, tutorials (class exercises) to implement the methods
  discussed in \texttt{R} will be provided.
\end{itemize}

\textbf{Assignment}

\begin{itemize}
\tightlist
\item
  10-page summary report on applying modern multivariate methods to
  particular problem areas, such as bioinformatics, finance, or
  marketing.
\end{itemize}

\textbf{Exam}

\begin{itemize}
\tightlist
\item
  \emph{Format}: Open-book take-home exam.
\item
  \emph{DP requirement}: lecture attendance and timely submission of
  tutorials.
\item
  \emph{Pass requirement}: Average of 50\% across exam and assignment,
  with 40\% sub-minimum for each.
\end{itemize}
\end{frame}

\begin{frame}{Assignment}
\protect\hypertarget{assignment}{}
\begin{itemize}
\tightlist
\item
  See
  \href{https://sebnemer.github.io/english/courses/multivariate/assignment-description.html}{course
  website} for details.
\item
  A template repository is available
  \href{https://github.com/MiguelRodo/Template24STA5069Z}{here} for
  creating a GitHub repo with the desired structure for the project.
\end{itemize}
\end{frame}

\begin{frame}{Technical prerequisites}
\protect\hypertarget{technical-prerequisites}{}
\begin{itemize}
\tightlist
\item
  Primarily, linear algebra. In particular:

  \begin{itemize}
  \tightlist
  \item
    Matrix norms
  \item
    Eigendecomposition
  \item
    SVD
  \end{itemize}
\end{itemize}
\end{frame}

\begin{frame}{References}
\protect\hypertarget{references}{}
\hypertarget{refs}{}
\begin{CSLReferences}{1}{0}
\leavevmode\vadjust pre{\hypertarget{ref-hastie_etal09}{}}%
Hastie, Trevor., Robert. Tibshirani, and J. H. (Jerome H.) Friedman.
2009. \emph{The Elements of Statistical Learning : Data Mining,
Inference, and Prediction}. 2nd ed. Springer Series in Statistics. {New
York}: {Springer}.

\leavevmode\vadjust pre{\hypertarget{ref-izenman08}{}}%
Izenman, Alan J. 2008. \emph{Modern {Multivariate Statistical
Techniques}}. Springer {Texts} in {Statistics}. {New York, NY}:
{Springer}. \url{https://doi.org/10.1007/978-0-387-78189-1}.

\end{CSLReferences}
\end{frame}



\end{document}
