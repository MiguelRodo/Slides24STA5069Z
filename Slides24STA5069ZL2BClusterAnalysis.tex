% Options for packages loaded elsewhere
\PassOptionsToPackage{unicode}{hyperref}
\PassOptionsToPackage{hyphens}{url}
\PassOptionsToPackage{dvipsnames,svgnames,x11names}{xcolor}
%
\documentclass[
  ignorenonframetext,
  aspectratio=169,
]{beamer}
\usepackage{pgfpages}
\setbeamertemplate{caption}[numbered]
\setbeamertemplate{caption label separator}{: }
\setbeamercolor{caption name}{fg=normal text.fg}
\beamertemplatenavigationsymbolsempty
% Prevent slide breaks in the middle of a paragraph
\widowpenalties 1 10000
\raggedbottom
\setbeamertemplate{part page}{
  \centering
  \begin{beamercolorbox}[sep=16pt,center]{part title}
    \usebeamerfont{part title}\insertpart\par
  \end{beamercolorbox}
}
\setbeamertemplate{section page}{
  \centering
  \begin{beamercolorbox}[sep=12pt,center]{part title}
    \usebeamerfont{section title}\insertsection\par
  \end{beamercolorbox}
}
\setbeamertemplate{subsection page}{
  \centering
  \begin{beamercolorbox}[sep=8pt,center]{part title}
    \usebeamerfont{subsection title}\insertsubsection\par
  \end{beamercolorbox}
}
\AtBeginPart{
  \frame{\partpage}
}
\AtBeginSection{
  \ifbibliography
  \else
    \frame{\sectionpage}
  \fi
}
\AtBeginSubsection{
  \frame{\subsectionpage}
}

\usepackage{amsmath,amssymb}
\usepackage{iftex}
\ifPDFTeX
  \usepackage[T1]{fontenc}
  \usepackage[utf8]{inputenc}
  \usepackage{textcomp} % provide euro and other symbols
\else % if luatex or xetex
  \usepackage{unicode-math}
  \defaultfontfeatures{Scale=MatchLowercase}
  \defaultfontfeatures[\rmfamily]{Ligatures=TeX,Scale=1}
\fi
\usepackage{lmodern}
\ifPDFTeX\else  
    % xetex/luatex font selection
\fi
% Use upquote if available, for straight quotes in verbatim environments
\IfFileExists{upquote.sty}{\usepackage{upquote}}{}
\IfFileExists{microtype.sty}{% use microtype if available
  \usepackage[]{microtype}
  \UseMicrotypeSet[protrusion]{basicmath} % disable protrusion for tt fonts
}{}
\makeatletter
\@ifundefined{KOMAClassName}{% if non-KOMA class
  \IfFileExists{parskip.sty}{%
    \usepackage{parskip}
  }{% else
    \setlength{\parindent}{0pt}
    \setlength{\parskip}{6pt plus 2pt minus 1pt}}
}{% if KOMA class
  \KOMAoptions{parskip=half}}
\makeatother
\usepackage{xcolor}
\newif\ifbibliography
\setlength{\emergencystretch}{3em} % prevent overfull lines
\setcounter{secnumdepth}{-\maxdimen} % remove section numbering


\providecommand{\tightlist}{%
  \setlength{\itemsep}{0pt}\setlength{\parskip}{0pt}}\usepackage{longtable,booktabs,array}
\usepackage{calc} % for calculating minipage widths
\usepackage{caption}
% Make caption package work with longtable
\makeatletter
\def\fnum@table{\tablename~\thetable}
\makeatother
\usepackage{graphicx}
\makeatletter
\def\maxwidth{\ifdim\Gin@nat@width>\linewidth\linewidth\else\Gin@nat@width\fi}
\def\maxheight{\ifdim\Gin@nat@height>\textheight\textheight\else\Gin@nat@height\fi}
\makeatother
% Scale images if necessary, so that they will not overflow the page
% margins by default, and it is still possible to overwrite the defaults
% using explicit options in \includegraphics[width, height, ...]{}
\setkeys{Gin}{width=\maxwidth,height=\maxheight,keepaspectratio}
% Set default figure placement to htbp
\makeatletter
\def\fps@figure{htbp}
\makeatother
\newlength{\cslhangindent}
\setlength{\cslhangindent}{1.5em}
\newlength{\csllabelwidth}
\setlength{\csllabelwidth}{3em}
\newlength{\cslentryspacingunit} % times entry-spacing
\setlength{\cslentryspacingunit}{\parskip}
\newenvironment{CSLReferences}[2] % #1 hanging-ident, #2 entry spacing
 {% don't indent paragraphs
  \setlength{\parindent}{0pt}
  % turn on hanging indent if param 1 is 1
  \ifodd #1
  \let\oldpar\par
  \def\par{\hangindent=\cslhangindent\oldpar}
  \fi
  % set entry spacing
  \setlength{\parskip}{#2\cslentryspacingunit}
 }%
 {}
\usepackage{calc}
\newcommand{\CSLBlock}[1]{#1\hfill\break}
\newcommand{\CSLLeftMargin}[1]{\parbox[t]{\csllabelwidth}{#1}}
\newcommand{\CSLRightInline}[1]{\parbox[t]{\linewidth - \csllabelwidth}{#1}\break}
\newcommand{\CSLIndent}[1]{\hspace{\cslhangindent}#1}

\usepackage{xcolor}
\usepackage{amssymb} % For additional symbols

% Definitions for all colors used in the presentation
\definecolor{balancedOrange}{RGB}{255,216,180}
\definecolor{secondaryOrange}{RGB}{251,188,5}
\definecolor{softBlue}{RGB}{100,124,156}
\definecolor{titleColor}{RGB}{0,51,102}
\definecolor{accentColor}{RGB}{112,173,71}

% Applying colors to Beamer elements
\setbeamercolor{palette primary}{bg=balancedOrange,fg=titleColor}
\setbeamercolor{palette secondary}{bg=secondaryOrange,fg=white}
\setbeamercolor{palette tertiary}{bg=softBlue,fg=white}
\setbeamercolor{titlelike}{parent=palette primary, fg=titleColor}
\setbeamercolor{structure}{fg=softBlue}
\setbeamercolor{block title}{bg=balancedOrange,fg=titleColor}
\setbeamercolor{block body}{bg=balancedOrange!10,fg=black}

% Customizing bullet points to a smaller sideways triangle and adjusting vertical position
\setbeamertemplate{itemize item}{\raisebox{1.5pt}{\scalebox{0.6}{$\blacktriangleright$}}}
\setbeamertemplate{itemize subitem}{\raisebox{1.3pt}{\scalebox{0.5}{$\blacktriangleright$}}}
\setbeamertemplate{itemize subsubitem}{\raisebox{1.1pt}{\scalebox{0.4}{$\blacktriangleright$}}}

% Suppressing the footline on the title slide
\setbeamertemplate{footline}{
    \ifnum\insertpagenumber=1
    \else
        \hfill\insertframenumber/\inserttotalframenumber\hspace*{1em}\vspace*{1em}
    \fi
}

% Must 
\makeatletter
\makeatother
\makeatletter
\makeatother
\makeatletter
\@ifpackageloaded{caption}{}{\usepackage{caption}}
\AtBeginDocument{%
\ifdefined\contentsname
  \renewcommand*\contentsname{Table of contents}
\else
  \newcommand\contentsname{Table of contents}
\fi
\ifdefined\listfigurename
  \renewcommand*\listfigurename{List of Figures}
\else
  \newcommand\listfigurename{List of Figures}
\fi
\ifdefined\listtablename
  \renewcommand*\listtablename{List of Tables}
\else
  \newcommand\listtablename{List of Tables}
\fi
\ifdefined\figurename
  \renewcommand*\figurename{Figure}
\else
  \newcommand\figurename{Figure}
\fi
\ifdefined\tablename
  \renewcommand*\tablename{Table}
\else
  \newcommand\tablename{Table}
\fi
}
\@ifpackageloaded{float}{}{\usepackage{float}}
\floatstyle{ruled}
\@ifundefined{c@chapter}{\newfloat{codelisting}{h}{lop}}{\newfloat{codelisting}{h}{lop}[chapter]}
\floatname{codelisting}{Listing}
\newcommand*\listoflistings{\listof{codelisting}{List of Listings}}
\makeatother
\makeatletter
\@ifpackageloaded{caption}{}{\usepackage{caption}}
\@ifpackageloaded{subcaption}{}{\usepackage{subcaption}}
\makeatother
\makeatletter
\@ifpackageloaded{tcolorbox}{}{\usepackage[skins,breakable]{tcolorbox}}
\makeatother
\makeatletter
\@ifundefined{shadecolor}{\definecolor{shadecolor}{rgb}{.97, .97, .97}}
\makeatother
\makeatletter
\makeatother
\makeatletter
\makeatother
\ifLuaTeX
  \usepackage{selnolig}  % disable illegal ligatures
\fi
\IfFileExists{bookmark.sty}{\usepackage{bookmark}}{\usepackage{hyperref}}
\IfFileExists{xurl.sty}{\usepackage{xurl}}{} % add URL line breaks if available
\urlstyle{same} % disable monospaced font for URLs
\hypersetup{
  pdftitle={Cluster analysis},
  pdfauthor={Miguel Rodo},
  colorlinks=true,
  linkcolor={blue},
  filecolor={magenta},
  citecolor={Blue},
  urlcolor={cyan},
  pdfcreator={LaTeX via pandoc}}

\title{Cluster analysis}
\subtitle{Clustering variables, assessing clustering algorithms and
biclustering}
\author{Miguel Rodo}
\date{2024-02-12}

\begin{document}
\frame{\titlepage}
\ifdefined\Shaded\renewenvironment{Shaded}{\begin{tcolorbox}[interior hidden, sharp corners, breakable, enhanced, boxrule=0pt, frame hidden, borderline west={3pt}{0pt}{shadecolor}]}{\end{tcolorbox}}\fi

\begin{frame}{Outline}
\protect\hypertarget{outline}{}
\begin{itemize}
\tightlist
\item
  Clustering variables using ClustOfVar
\item
  Clustering observationals and variables using biclustering
\end{itemize}
\end{frame}

\begin{frame}[fragile]{M. Chavent et al. (2011)}
\protect\hypertarget{chavent_etal11}{}
\begin{figure}[H]
\centering
\includegraphics[width=0.75\textwidth]{_data_raw/img/clust_of_var.png}
\end{figure}

\begin{itemize}
\tightlist
\item
  The algorithm \texttt{ClustOfVar} clusters a mix of quantitative and
  qualitative variables
\end{itemize}
\end{frame}

\begin{frame}{Synthetic variables}
\protect\hypertarget{synthetic-variables}{}
\begin{itemize}
\tightlist
\item
  Clustering may be both hierarchical and non-hierarchical (k-means)
\item
  The key novelty is the introduction of a synthetic variable for each
  cluster, used to guide clustering

  \begin{itemize}
  \tightlist
  \item
    For a given cluster, the synthetic variable is the first principal
    component of the variables in the cluster
  \item
    As the data may have qualitative and quantitative variables, the
    algorithm PCAMix (Marie Chavent, Kuentz-Simonet, and Saracco 2012)
    is used.
  \end{itemize}
\end{itemize}
\end{frame}

\begin{frame}{Homogeneity}
\protect\hypertarget{homogeneity}{}
\begin{itemize}
\tightlist
\item
  The synthetic variable is used to define the homogeneity
  (``togetherness'') of a cluster
\item
  Definition of homogeneity:

  \begin{itemize}
  \tightlist
  \item
    Sum of \(R^2\) between the synthetic variable and each variable in
    the cluster
  \item
    \(R^2\) is the sum proportion of variation in the dependent variable
    (the synthetic variable in this case) explained by the independent
    variables (the variables in the cluster)
  \end{itemize}
\end{itemize}
\end{frame}

\begin{frame}{Hierarchical clustering algorithm}
\protect\hypertarget{hierarchical-clustering-algorithm}{}
\begin{itemize}
\tightlist
\item
  An agglomerative hierarchical clustering procedure is employed
\item
  The choice of which two clusters to merge is based on the homogeneity
  of the original and resulting clusters. This is the only novelty.
\item
  Specifically, for \(H(\cdot)\) the homogeneity of a given cluster and
  \(A\) and \(B\) two clusters, the algorithm merges two clusters such
  that \(d(A, B) = H(A) + H(B) - H(A\cup B)\) is minimised.
\end{itemize}
\end{frame}

\begin{frame}{Partitioning algorithm}
\protect\hypertarget{partitioning-algorithm}{}
\begin{itemize}
\tightlist
\item
  As with hierarchical clustering, the partitioning algorithm uses the
  synthetic variables to guide the clustering.
\item
  For a given cluster with synthetic variable, the association with an
  actual variable is measured by the canonical correlation coefficient

  \begin{itemize}
  \tightlist
  \item
    As we are only considering the first canonical variate and the
    synthetic variable is quantitative (not categorical), this is equal
    to the \(R^2\) of a linear regression of the synthetic variable on
    the actual varialble
  \end{itemize}
\item
  As before, variables are allocated to clusters for which the
  dissimilarity is minimised (canonical correlation with synthetic
  variable is maximised).
\end{itemize}
\end{frame}

\begin{frame}{Choosing the number of clusters}
\protect\hypertarget{choosing-the-number-of-clusters}{}
\begin{columns}[T]
\begin{column}{0.5\textwidth}
\begin{block}{Cluster stability}
\protect\hypertarget{cluster-stability}{}
\begin{itemize}
\tightlist
\item
  Assessed using cluster stability under resampling
\item
  Essentially, this is the procedure:

  \begin{itemize}
  \tightlist
  \item
    Bootstrap \(B\) samples of the \(n\) observations
  \item
    Apply the clustering algorithm to each bootstrap sample
  \item
    Calculate the Rand index (Rand 1971) between the clusters obtained
    from the original sample and the clusters obtained from the
    bootstrap sample
  \end{itemize}
\item
  The average Rand index is the cluster stability.
\end{itemize}
\end{block}
\end{column}

\begin{column}{0.5\textwidth}
\begin{block}{Rand index}
\protect\hypertarget{rand-index}{}
\begin{itemize}
\tightlist
\item
  The Rand index is a rather odd measure to assess the similarity of the
  two clusters.

  \begin{itemize}
  \tightlist
  \item
    For a given pair observations, they are regarded as having been
    clustered the same way if they either are clustered together in both
    clusterings or are clustered separately in both clusterings.
  \item
    The Rand index is the proportion of pairs of observations that are
    clustered the same way in both clusterings.
  \end{itemize}
\item
  The adjusted Rand index (Hubert and Arabie 1985) corrects for chance
  agreement.
\end{itemize}
\end{block}
\end{column}
\end{columns}
\end{frame}

\begin{frame}{Assessment of \texttt{ClustOfVar}}
\protect\hypertarget{assessment-of-clustofvar}{}
\begin{itemize}
\tightlist
\item
  Essentially performing regression or the SVD each time the
  dissimilarity needs to be calculated is time-consuming.

  \begin{itemize}
  \tightlist
  \item
    They note that performance is slow when there are many variables.
  \item
    At least at the time of writing, a parallel version of the algorithm
    was planned.
  \item
    Looking at their GitHub repository, this does not seem to have been
    done.

    \begin{itemize}
    \tightlist
    \item
      Package is on CRAN, but has not been updated in years.
    \end{itemize}
  \end{itemize}
\item
  Good that they made an attempt to help guide the number of clusters
  selected

  \begin{itemize}
  \tightlist
  \item
    Unusual that they ignored per-cluster stability assessments (Hennig
    2007)
  \end{itemize}
\item
  Appropriateness in particular domains would need to be asssessed
  (i.e.~how does it do on particular kinds of datasets, e.g.~economic,
  biological)
\end{itemize}
\end{frame}

\begin{frame}[fragile]{Selection of clustering algorithms}
\protect\hypertarget{selection-of-clustering-algorithms}{}
\begin{itemize}
\tightlist
\item
  Part of the purpose of showing \texttt{ClustOfVar} is to show the
  flexibility in coming up with new clustering algorithms

  \begin{itemize}
  \tightlist
  \item
    The first set of slides introduce basic clustering approaches
  \item
    These may be combined with other techniques (such as PCAMix) to
    create new algorithms
  \end{itemize}
\item
  The problme afterwards is - which algorithm to use?
\item
  Typically, there are two main considerations:

  \begin{itemize}
  \tightlist
  \item
    Theoeretical considerations

    \begin{itemize}
    \tightlist
    \item
      For example: computational complexity (affecting run time, memory
      constraints), assumptions about the data (e.g.~normality), whether
      the number of clusters is pre-specified, etc.
    \end{itemize}
  \item
    Empirical considerations

    \begin{itemize}
    \tightlist
    \item
      Performance in real world datasets: correspondence with manual
      labels, stability, ability to identify predictive variables, etc.
    \end{itemize}
  \end{itemize}
\end{itemize}
\end{frame}

\begin{frame}{Labelling cells}
\protect\hypertarget{labelling-cells}{}
\begin{itemize}
\tightlist
\item
  Modern experimental techniques measure tens of variables on millions
  of cells rapidly
\item
  The cells need to be labelled, e.g.~as T cells, B cells, etc., which
  first requires clustering them.
\item
  Traditionally, this was done by hand (as below), but this is very slow
  at scale:
\end{itemize}

\begin{figure}[H]
\centering
\includegraphics[width=0.675\textwidth]{_data_raw/img/manual_gating.png}
\caption{Finak (2014)}
\end{figure}
\end{frame}

\begin{frame}{Aghaeepour et al. (2013)}
\protect\hypertarget{aghaeepour_etal13}{}
\begin{itemize}
\tightlist
\item
  Whilst manual clustering of cells is slow, it is relatively trusted.
\item
  It was not clear how well automated algorithms would perform, in terms
  of reproducing manual clusterings or identifying biologically relevant
  clusters.
\item
  Aghaeepour et al. (2013) therefore constructed an empirical
  comparison, assessing algorithm performance on multiple datasets in
  terms of the following criteria:

  \begin{itemize}
  \tightlist
  \item
    Ability to reproduce manual clusterings
  \item
    Ability to identify cell types associated with disease
  \end{itemize}
\end{itemize}
\end{frame}

\begin{frame}{Several automated algorithms typically identified manual
clusters well}
\protect\hypertarget{several-automated-algorithms-typically-identified-manual-clusters-well}{}
\begin{figure}[H]
\centering
\includegraphics[width=1\textwidth]{_data_raw/img/perf-cell_id.png}
\caption{Aghaeepour (2013)}
\end{figure}
\end{frame}

\begin{frame}{Biclustering}
\protect\hypertarget{biclustering}{}
\begin{itemize}
\tightlist
\item
  Goal is to identify subgroups of observations and variables that are
  highly correlated
\item
  For example:

  \begin{itemize}
  \tightlist
  \item
    In gene expression data, we may want to identify genes that are
    co-expressed in a subset of samples

    \begin{itemize}
    \tightlist
    \item
      Several genes may only be expressed (made/increased/elevated) in
      patients with, say, flu, but these genes are not expressed by
      healthy individuals or patients with other diseases
    \end{itemize}
  \item
    Attempting to cluster genes and samples separately may miss these
    patterns
  \end{itemize}
\end{itemize}
\end{frame}

\begin{frame}{ANOVA model for biclustering I}
\protect\hypertarget{anova-model-for-biclustering-i}{}
\begin{itemize}
\tightlist
\item
  We assume that the expression level of gene \(i\) in sample \(j\) is
  given by:
\end{itemize}

\[
y_{ij} = \mu + \alpha_i + \beta_j + \epsilon_{ij}
\]

\begin{itemize}
\tightlist
\item
  where \(\mu\) is average expression level, \(\alpha_i\) is the effect
  of gene \(i\), \(\beta_j\) is the effect of sample \(j\), and
  \(\epsilon_{ij}\) is the error term.

  \begin{itemize}
  \tightlist
  \item
    Note that, in this case, samples are along the columns and variables
    along the rows.
  \end{itemize}
\item
  A cluster is a subset of genes and samples for which the \(\alpha_i\)
  and \(\beta_j\) are similar.
\end{itemize}
\end{frame}

\begin{frame}{ANOVA model for biclustering II}
\protect\hypertarget{anova-model-for-biclustering-ii}{}
\begin{itemize}
\tightlist
\item
  We aim to find a set of genes \(I\) and set of samples \(J\) such that
\end{itemize}

\[
\sum_{i\in I}\sum_{j\in J} \hat{\epsilon}_{ij}^2=\sum_{i\in I}\sum_{j\in J} (y_{ij} - \hat{\mu} - \hat{\alpha}_i - \hat{\beta}_j)^2
\]

\begin{itemize}
\tightlist
\item
  is minimised. To account for varying numbers of samples, we divide by
  the product of the number of samples and number of genes.
\item
  A greedy algorithm is used to find the biclusters (delete/add
  rows/columns to increase objective function until no improvement
  possible). Subsequent biclusters found in the same way, but after
  replacing rows/columns in the current bicluster with random numbers.
\end{itemize}
\end{frame}

\begin{frame}[fragile]{\texttt{R} example of biclustering}
\protect\hypertarget{r-example-of-biclustering}{}
\begin{itemize}
\tightlist
\item
  Available in the \texttt{Cluster\ Analysis\ (2)} slides from last
  year.
\end{itemize}
\end{frame}

\begin{frame}[allowframebreaks]{Complete references}
\protect\hypertarget{complete-references}{}
\hypertarget{refs}{}
\begin{CSLReferences}{1}{0}
\leavevmode\vadjust pre{\hypertarget{ref-aghaeepour_etal13}{}}%
Aghaeepour, Nima, Greg Finak, Holger Hoos, Tim R. Mosmann, Ryan
Brinkman, Raphael Gottardo, and Richard H. Scheuermann. 2013.
{``Critical Assessment of Automated Flow Cytometry Data Analysis
Techniques.''} \emph{Nature Methods} 10 (3, 3): 228--38.
\url{https://doi.org/10.1038/nmeth.2365}.

\leavevmode\vadjust pre{\hypertarget{ref-chavent_etal12}{}}%
Chavent, Marie, Vanessa Kuentz-Simonet, and Jérôme Saracco. 2012.
{``Orthogonal Rotation in {PCAMIX}.''} \emph{Advances in Data Analysis
and Classification} 6 (2): 131--46.
\url{https://doi.org/10.1007/s11634-012-0105-3}.

\leavevmode\vadjust pre{\hypertarget{ref-chavent_etal11}{}}%
Chavent, M., V. Kuentz, B. Liquet, and L. Saracco. 2011.
{``{ClustOfVar}: {An R Package} for the {Clustering} of {Variables}.''}
December 1, 2011. \url{https://doi.org/10.48550/arXiv.1112.0295}.

\leavevmode\vadjust pre{\hypertarget{ref-finak_etal14d}{}}%
Finak, Greg, Wenxin Jiang, Kevin Krouse, Chungwen Wei, Ignacio Sanz,
Deborah Phippard, Adam Asare, Stephen C. De Rosa, Steve Self, and
Raphael Gottardo. 2014. {``High-Throughput Flow Cytometry Data
Normalization for Clinical Trials.''} \emph{Cytometry Part A} 85 (3):
277--86. \url{https://doi.org/10.1002/cyto.a.22433}.

\leavevmode\vadjust pre{\hypertarget{ref-hennig07}{}}%
Hennig, Christian. 2007. {``Cluster-Wise Assessment of Cluster
Stability.''} \emph{Computational Statistics \& Data Analysis} 52 (1):
258--71. \url{https://doi.org/10.1016/j.csda.2006.11.025}.

\leavevmode\vadjust pre{\hypertarget{ref-Hubert1985}{}}%
Hubert, Lawrence, and Phipps Arabie. 1985. {``Comparing Partitions.''}
\emph{Journal of Classification} 2 (1): 193--218.
\url{https://EconPapers.repec.org/RePEc:spr:jclass:v:2:y:1985:i:1:p:193-218}.

\leavevmode\vadjust pre{\hypertarget{ref-rand71}{}}%
Rand, William M. 1971. {``Objective {Criteria} for the {Evaluation} of
{Clustering Methods}.''} \emph{Journal of the American Statistical
Association} 66 (336): 846--50. \url{https://doi.org/10.2307/2284239}.

\end{CSLReferences}
\end{frame}



\end{document}
